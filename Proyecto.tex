
%Begin Preambulo
\documentclass[12pt,letterpaper,openright]{article}
%---------------------------------------------------------------------------
%Packages
%Paquete para matematicas
 \usepackage{amsmath}
%Paquete para la codificacion de simbolos
\usepackage[utf8]{inputenc}
%Paquete para los comentarios
\usepackage{comment}
%Paquete para el idioma de las etiquetas que latex mostrara en el documento
\usepackage[spanish]{babel}
%Paquete para bibliografía
\usepackage{cite} 
%Paquete espacios entre titulos
\usepackage{titlesec}
%Paquete para imagenes
\usepackage{graphicx} % graficos
 \graphicspath{ {images/} }%path de imagenes
%Paquete para ver vinculos
\usepackage{hyperref}
\hypersetup{
    colorlinks=true,
    linkcolor=blue,
    filecolor=magenta,      
    urlcolor=cyan,
    pdftitle={Herramientas para Analizador.},
    bookmarks=true,
    %pdfpagemode=FullScreen,
}
%paquete enumerados referencias
\usepackage{enumerate} 
%Encambezados
%Paquete para emcabezados
\usepackage{fancyhdr}
\pagestyle{fancy}
\fancyhf{}
\rhead{\begin{picture}(0,0) \put(-56.7,0){\includegraphics[width=20mm]{images/FACUING.png}} \end{picture}}
\chead{Universidad Autónoma del estado de México.\\ Valle de México.}
%Share\LaTeX
\lhead{\begin{picture}(0,0) \put(0,0){\includegraphics[width=20mm]{images/UAEM.jpg}} \end{picture}}
\cfoot{Ingeniería en Computación.}
\rfoot{Pagina \thepage}
\renewcommand{\headrulewidth}{2pt}
\renewcommand{\footrulewidth}{1pt}

%---------------------------------------------------------------------------
%---------------------------------------------------------------------------
\title{\rule{14cm}{0.1mm}\\ Compiladores \\ }
\author{Noé Vásquez Godínez  \\ Ulises Rodrigo Osnaya  \\ 
       \rule{14cm}{0.1mm}
       }
\date{7 Marzo 2018}
                 %izquierda  arriba          abajo   zquierda
                 
\titlespacing{\section}{1pc}{1ex plus .1ex minus 1ex}{1pc}

%End Preambulo 
%---------------------------------------------------------------------------
%---------------------------------------------------------------------------
\begin{document}
	\begin{titlepage}
		\maketitle
		\begin{figure}[h]
		%\caption{Example of a parametric plot ($\sin (x), \cos(x), x$)}
		\includegraphics[scale=2]{logo.jpg}
		\centering
		\end{figure}
		\center Realizado en \LaTeX
\end{titlepage}

\begin{abstract}
\end{abstract}
\newpage
\thispagestyle{empty}
\tableofcontents{}
\newpage
\thispagestyle{empty}
\listoffigures
\newpage
\addcontentsline{toc}{section}{Introducción.}
\newpage

\section{Introducción.}
\newpage
\section{La computadora y los lenguajes de programación.}
	\subsection{La computadora}
		\subsubsection{Breve historia del origen de la computadora.}
		\subsubsection{Arquitectura de la computadora.}
	\subsection{El procesador.}
		\subsubsection{Lenguaje Ensamblador.}
	\subsection{Lenguajes de programación.}
\newpage	
\section{Teoría de la computación.}
	 \subsection{Matemáticas.}
	 \subsection{Autómata Finito.}
	 \subsection{Lenguajes regulares y gramáticas regulares.}
	 \subsection{Propiedades lenguajes regulares.}
	 \subsection{Lenguajes libres de contexto.}
	 \subsection{Autómata de pila.}
	 \subsection{Propiedades de los lenguajes libres de contexto.}
\newpage
\section{Compiladores.}
	\subsection{Estructura de un compilador}
		\subsubsection{Analizador Léxico.}
		\subsubsection{Analizador Sintáctico.}
		\subsubsection{Analizador Semántico.}
		\subsubsection{Generación de código.}
		
\newpage
\section{Programando un compilador.}

%\bibliographystyle{plain}
%\bibliography{biblio.bib}
\end{document}

